\documentclass{llncs}
\usepackage{authblk}

\title{Mobile Alternative of the MoneyBee Project for Financial Forecasting}

\author{Petar Tomov, Iliyan Zankinski, Maria Barova}

\institute{Institute of Information and Communication Technologies \\
Bulgarian Academy of Sciences \\
acad. Georgi Bonchev Str, block 2, office 514, 1113 Sofia, Bulgaria \\
\email{p.tomov@iit.bas.bg} \\
\texttt{http://www.iict.bas.bg/}}

% Petar Tomov p.tomov@iit.bas.bg 
% Iliyan Zankinski iliyan@hsi.iccs.bas.bg
% Maria Barova m.barova@iit.bas.bg

\date{January 28, 2018}

\begin{document}

\maketitle

ABSTRACT

\vspace{3mm}

Donated distributed computing is pretty popular in the last two decades. According to DistributedComputing.Info website, there are hundreds of distributed computing projects in many scientific areas. The most interesting fact, clearly visible in the website, is that there is only one active financial project called MQL5 Cloud Network. Between 2000 and 2010 there was a German project called MoneyBee. The aim of this project was financial forecasting by artificial neural networks which were trained in donated distributed computing environment. MoneyBee project was developed and operated by i42 Informationsmanagement GmbH company. It was a distributed solution based on central server and many remote workers. Calculations on the workers were done during the desktop idle mode when screen saver was running. In fact the MoneyBee on the client machines was the screensaver itself. The goal of current research is to recreate the ideas used in MoneyBee project, but with the usage of the modern mobile devices. For the needs of the research two GitHub projects are used - VitoshaTrade and VitDisComp.

\end{document}